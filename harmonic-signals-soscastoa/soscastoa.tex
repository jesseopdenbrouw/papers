\documentclass[12pt,fleqn]{article}

\usepackage[a4paper,left=1in,right=1in,top=1in,bottom=1.5in]{geometry}
\usepackage[charter]{mathdesign}
\usepackage{parskip}
\usepackage{mathtools}
%\usepackage{amssymb}
\setlength{\mathindent}{1em}
\newcommand\strutrule{\rule{0mm}{1.9ex}}
\DeclareMathSymbol{,}{\mathord}{letters}{"3B}
\usepackage{microtype}
\usepackage{float}
\usepackage{xcolor}
\usepackage{tikz}
\usetikzlibrary{decorations.pathreplacing,calc}
\usepackage{pgfplots}
\usepgfplotslibrary{fillbetween}
\pgfplotsset{compat=1.16}

\begin{document}

\begin{center}
\huge SOS-CAS-TOA

\large De sinus, cosinus en tangens van een hoek
\end{center}

\vspace*{1cm}

In een rechthoekige driehoek ABC gelden de sinus- cosinus- en tangensregels.

\begin{figure}[!h]
\centering
\begin{tikzpicture}[scale=1.5,thick]
\draw (0,0) node (A) [left] {A} -- (36.869:5) node (B) [above] {B} node[midway,above] {5} -- (4,0) node (C) [right] {C} node[midway, right] {3} -- cycle node[midway,below] {4};
\draw[<->] (0:1) arc (0:36.869:1) node [midway,right] {$a$};
\draw[thin] (3.8,0) |- (4,0.2);
\end{tikzpicture}
\end{figure}

SOS: Sinus is Overstaande zijde gedeeld door de Schuine zijde:
\begin{equation}
\boxed{\sin a = \dfrac{BC}{AB}}
\end{equation}

CAS: Cosinus is Aanliggende zijde gedeeld door de Schuine zijde:
\begin{equation}
\boxed{\cos a = \dfrac{AC}{AB}}
\end{equation}

TOA: Tangens is Overstaande zijde gedeeld door de Aanliggende zijde:
\begin{equation}
\boxed{\tan a = \dfrac{BC}{AC}}
\end{equation}

Pythagoras:
\begin{equation}
\boxed{AB^2 = AC^2 + BC^2}
\end{equation}

Voor de sinus, cosinus en tangens in de figuur gelden:
\begin{equation}
\jot=8pt
\begin{split}
\sin a &= \dfrac{BC}{AB} = \dfrac{3}{5} = 0,6 \\
\cos a &= \dfrac{AC}{AB} = \dfrac{4}{5} = 0,8 \\
\tan a &= \dfrac{BC}{AC} = \dfrac{3}{4} = 0,75
\end{split}
\end{equation}

Dezelfde waarden voor de sinus, cosinus en tangens worden gevonden als we de driehoek vergroten of verkleinen met een vergrotingsfactor.

Voor de tangens geldt ook:
\begin{equation}
\tan a = \dfrac{BC}{AC} = \dfrac{\dfrac{BC}{AB}}{\dfrac{AC}{AB}} \qquad\text{met }\dfrac{BC}{AB}=\sin a\text{ en }\dfrac{AC}{AB} = \cos a
\end{equation}
%
Dus geldt:
%
\begin{equation}
\boxed{\tan a = \dfrac{\sin a}{\cos a}}
\end{equation}

Voor het vinden van een hoek moeten de inverse bewerkingen gedaan worden:
\begin{equation}
\begin{split}
a &= \sin^{-1} 0,6 = 36,87^{\circ} \\
a &= \cos^{-1} 0,8 = 36,87^{\circ} \\
a &= \tan^{-1} 0,75 = 36,87^{\circ} \\
\end{split}
\end{equation}

\subsubsection*{Notatie}
Een veel gebruikte notatie voor het berekenen van een hoek vanuit de sinus en cosinus is $\sin^{-1}$ en $\cos^{-1}$.
Dat is formeel gezien niet juist. De afspraak is om een sinus-tot-de-macht aan te geven met $\sin^{macht}$. Dus dit zou betekenen dat:
%
\begin{equation}
\text{sinus-tot-de-macht-min-1} = \sin^{-1} = \dfrac{1}{\sin}
\end{equation}
%
Dit geldt ook zo voor cosinus en tangens. De officiële notatie is \textsl{arcsinus}, \textsl{arccosinus} en \textsl{arctangens}:
%
\begin{equation}
a = \arcsin x \qquad a = \arccos x \qquad a = \arctan x
\end{equation}
%
Voor het \textsl{kwadraat} (en hogere machten) kunnen we de macht bij de sinus, cosinus of tangens schrijven:
%
\begin{equation}
\boxed{(\sin a)\cdot(\sin a) = \sin^2 a}
\end{equation}

\subsubsection*{Cirkel en driehoek}
We kunnen een driehoek ook tekenen in een cirkel. De cirkel tekenen we in een xy-vlak. We nemen voor het gemak een cirkel met de straal 1. Dit wordt de eenheidscirkel genoemd. De driehoek $OBC$ wordt zo getekend dat punt $O$ op de oorsprong ligt én het middelpunt is van de cirkel. Punt $B$ ligt onder een hoek $a$ op de cirkel. Punt $C$ ligt loodrecht onder punt $B$ op de x-as. Lijnstukken $OC$ en $BC$ vormen dus een rechte hoek. Zie onderstaande figuur.

\begin{figure}[!h]
\centering
\begin{tikzpicture}[scale=0.9,thick]
\draw[-,dashed] (0,-5.5) -- (0,5.5) node [above] {$y$};
\draw[-,dashed] (-5.5,0) -- (5.5,0) node [right] {$x$};
\draw (0,0) circle (5);
\draw (0,0) node (A) [left,fill=white] {O} -- (36.869:5) node (B) [above,right] {B} node[midway,above] {1} -- (4,0) node (C) [right,fill=white] {C} node[midway, right] {0,6} -- cycle node[midway,below] {0,8};
\draw[<->] (0:1) arc (0:36.869:1) node [midway,right] {$a$};
\draw[thin] (3.8,0) |- (4,0.2);
\end{tikzpicture}
\end{figure}

Omdat de schuine zijde nu de lengte 1 heeft, vereenvoudigen de functies voor sinus en cosinus.
%
\begin{equation}
\jot=8pt
\begin{split}
\sin a &= \dfrac{BC}{OB} = \dfrac{0,6}{1} = 0,6 \\
\cos a &= \dfrac{OC}{OB} = \dfrac{0,8}{1} = 0,8 \\
\tan a &= \dfrac{BC}{OC} = \dfrac{0,6}{0,8} = 0,75
\end{split}
\end{equation}

Ook nu geldt de stelling van Pythagoras:
\begin{equation}
OB^2 = OC^2 + BC^2
\end{equation}

Dus geldt:
\begin{equation}
1^2 = 0,6^2 + 0,8^2
\end{equation}

Maar de waarde $0,6$ volgt uit $\sin a$ en de waarde $0,8$ volgt uit $\cos a$. Dus moet gelden:
\begin{equation}
\boxed{\sin^2 a + \cos^2 a = 1}
\end{equation}
%
Dit wordt de \textsl{hoofdstelling van de goniometrie} genoemd.

De waarden van de sinus, cosinus en tangens kunnen ook negatief zijn. Dit is te zien in de onderstaande figuur. Zo is $\sin 215^\circ \approx -0,5736$ en $\cos 215^\circ \approx -0,8191$. 

\begin{figure}[H]
\centering
\begin{tikzpicture}[scale=0.666,thick]
\draw[-,dashed] (0,-5.5) -- (0,5.5) node [above] {$y$};
\draw[-,dashed] (-5.5,0) -- (5.5,0) node [right] {$x$};
\draw[->] (0:1) arc (0:215:1) node [midway,above] {$a$};
\draw[-latex] (0,0) -- (215:5) node [midway,above right] {};
\draw[dashed] (215:5) -| (0,0);
\node at (0,{5*sin(215)}) [right] {$\sin a$};
\draw[dashed] (215:5) |- (0,0);
\node at ({5*cos(215)},0)[above] {$\cos a$};
\end{tikzpicture}
\end{figure}

 
Als we in een xy-vlak een punt $P$ aangeven met de x-coördinaat afhankelijk van de cosinus en de y-coördinaat van de sinus van de hoek, en we laten de hoek oplopen van $0^\circ$ tot en met $360^\circ$, dan krijgen we een \textsl{cirkel}. De straal van de cirkel is~1. Dus geldt:
%
\begin{equation}
P(x, y) = \begin{dcases}
x = \cos a \\
y = \sin a
\end{dcases}
\end{equation}

\begin{figure}[H]
\centering
\begin{tikzpicture}[scale=0.666,thick]
\draw[-,dashed] (0,-5.5) -- (0,5.5) node [above] {$y$};
\draw[-,dashed] (-5.5,0) -- (5.5,0) node [right] {$x$};
%\draw[dashed] (0,0) circle (5);
\draw[<->] (0:5) arc (0:330:5) node [right] {$P$};
\draw[dashed,-latex] (0,0) -- (330:5) node [midway,above right] {1};
\end{tikzpicture}
\end{figure}

\subsubsection*{Grafieken sinus en cosinus}
Als we de waarde van de sinus uitzetten (op de y-as) als functie van de hoek (op de x-as) dan krijgen we een zogenoemde \textsl{golfvorm}:

\begin{tikzpicture}% function
\begin{axis}[
	xlabel=$a$,
	ylabel=$\sin a$,
	width=0.9\textwidth,
	height=0.4\textwidth,
	xmin=0,
	xmax=360,
	xtick={0,45,...,360},
	axis y line*=left,
	axis x line*=middle,
	x axis line style={-},
	y axis line style={-},
	axis line style = thick,
	samples=101
]
\addplot[domain=0:360,red] {sin(x)};
\addplot[domain=0:360,blue,loosely dashed] {1};
\addplot[domain=0:360,blue,loosely dashed] {-1};
\end{axis}
\end{tikzpicture}

Als we de waarde van de cosinus uitzetten (op de y-as) als functie van de hoek (op de x-as) dan krijgen we een zogenoemde \textsl{golfvorm}:

\begin{tikzpicture}% function
\begin{axis}[
	xlabel=$a$,
	ylabel=$\cos a$,
	width=0.9\textwidth,
	height=0.4\textwidth,
	xmin=0,
	xmax=360,
	xtick={0,45,...,360},
	axis y line*=left,
	axis x line*=middle,
	x axis line style={-},
	y axis line style={-},
	axis line style = thick,
	samples=101
]
\addplot[domain=0:360,red] {cos(x)};
\addplot[domain=0:360,blue,loosely dashed] {1};
\addplot[domain=0:360,blue,loosely dashed] {-1};
\end{axis}
\end{tikzpicture}

De sinus en cosinus zijn zogenoemde periodieke functies: ze herhalen zich om de $360^\circ$. De sinus en cosinus zijn zogenoemde harmonische signalen. Dit zijn signaalvormen die voorkomen in veel natuurkundige processen.

Enkele formules:
%
\begin{equation}
\begin{split}
\sin a &= \sin(a+360^\circ) \\
\cos a &= \cos(a+360^\circ) \\
\tan a &= \tan(a+180^\circ)
\end{split}\end{equation}
%
Verder is te zien dat de cosinus en sinus $90^\circ$ verschoven zijn:
\begin{equation}\label{key}
\begin{split}
\cos a &= \sin (a+90^\circ) \\
\sin a &= \cos (a-90^\circ)
\end{split}
\end{equation}
%De tangens loopt op van 0 tot oneindig.
%
%\begin{tikzpicture}% function
%\begin{axis}[
%	xlabel=$a$,
%	ylabel=$\tan a$,
%	width=0.9\textwidth,
%	height=0.4\textwidth,
%	xmin=0,
%	xmax=360,
%	restrict y to domain=-20:20,
%	xtick={0,45,...,360},
%	axis y line*=left,
%	axis x line*=middle,
%	x axis line style={-},
%	y axis line style={-},
%	axis line style = thick,
%	samples=201
%]
%\addplot[domain=0:360,red] {tan(x)};
%\addplot[domain=0:360,blue,loosely dashed] {1};
%\addplot[domain=0:360,blue,loosely dashed] {-1};
%\end{axis}
%\end{tikzpicture}

\subsubsection*{Stijgingspercentage en hellingshoek}
De stijgingspercentage wordt aangegeven in procenten, bijvoorbeeld 14\%.

\begin{figure}[!h]
\centering
\begin{tikzpicture}[scale=1.5,thick]
\draw (0,0) {} -- (5,1) node[midway,above] {C} -- (5,0) node (C) {} node[midway, right] {B} -- cycle node[midway,below] {A} (C);
\draw[<->] (0:2) arc (0:11.3:2) node [midway,right] {$a$};
\draw[thin] (4.8,0) |- (5,0.2);
\end{tikzpicture}
\end{figure}

\begin{equation}
\text{stijgingspercentage} = 100\% \times \dfrac{B}{A}
\end{equation}
en
\begin{eqnarray}
\text{stijgingspercentage} = 100\% \times \tan a
\end{eqnarray}

De hellingshoek is:
\begin{equation}
a = \tan^{-1} \dfrac{\text{stijgingspercentage}}{100\%}
\end{equation}

Voorbeeld:

Stijgingspercentage is 14\%. Dan is de hoek:
\begin{equation}
a = \tan^{-1} \dfrac{14\%}{100\%} = \tan^{-1} 0,14 = 8^\circ
\end{equation}

Voorbeeld:

Een skipiste is 2 km lang en overbrugt een hoogteverschil van 100 m. Dus $C=2000$ en $B=100$. Nu de hoek $a$ berekenen:
\begin{equation}
a = \sin^{-1} \dfrac{B}{C} = \sin^{-1} \dfrac{100}{2000} = 2,866^\circ
\end{equation}

Het stijgingspercentage is dan:
%
\begin{equation}
\text{stijgingspercentage} = 100\% \times \tan a = 100\% \times \tan 2,886 = 5\%
\end{equation}
\end{document}