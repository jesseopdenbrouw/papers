\documentclass[a4paper,12pt,oneside]{article}

\author{Jesse op den Brouw}
\title{Overflow-detectie in de 2's complement representatie}
\date{\today}

\usepackage[dutch]{babel}

%% Making captions nicer...
\usepackage[font=footnotesize,format=plain,labelfont=bf,up,textfont=it,up]{caption}
\captionsetup[table]{justification=centering,singlelinecheck=false,skip=0pt}
\captionsetup[figure]{belowskip=-0.5\baselineskip}

%\def\Hrulefill{\vspace{-\baselineskip}\leavevmode\leaders\hrule height 0.1pt \hfill\kern0pt\relax} 

%% Use the AMS Mathematical characters
\usepackage{mathtools}
\usepackage{amsfonts}
\usepackage{amssymb}
%\setlength{\mathindent}{1em}
\DeclareMathSymbol{,}{\mathord}{letters}{"3B}
%% A bit nicer bar / overline command
\newcommand*{\oline}[1]{\overline{#1\mathstrut}}
\usepackage{amsthm}

\newtheoremstyle{own}%
    {0.3\baselineskip}% Space above
    {-10pt}% Space below
    {}% Body font
    {}% Indent amount
    {\bfseries}% Theorem head font
    {.}% Punctuation after theorem head
    {\newline}% Space after theorem head
    {}% Theorem head spec

\theoremstyle{own}
\newtheorem{example}{Voorbeeld}[section]

\usepackage{float}
\usepackage{booktabs}
%\usepackage{parskip}
\setlength{\parindent}{0pt}
\setlength{\parskip}{0.7\baselineskip}

\usepackage{array}
\usepackage{pict2e}
\usepackage{calc}
\usepackage{color}

\usepackage[stretch=10]{microtype}


\begin{document}
\maketitle

\vfill

%\begin{abstract}
%In de digitale techniek worden rekenschakelingen ontworpen die gebruik maken
%van de 2's complement representatie. Hierin worden positieve en negatieve
%getallen voorgesteld
%Dat levert eenvoudige, compacte en snelle hardware op.
%De keerzijde is dat het voor mensen niet eenvoudig is om (de waarde van) de
%getallen direct te herkennen.
%
%Daarnaast levert het gebruik van de 2's complement representatie nog een
%probleem om: het resultaat van een bewerking kan buiten het bereik van de
%representatie liggen. Er is dan sprake van overflow.
%
%Het detecteren van een overflow is een vereiste om de berekeningen in goede
%banen te laten lopen. Natuurlijk is de vraag: wat moet er gebeuren als er
%overflow geconstateerd is.
%\end{abstract}

\newpage
\section{Introductie}
In de digitale techniek worden rekenschakelingen ontworpen die gebruik maken
van de 2's complement representatie. Hierin worden positieve en negatieve
getallen uit een bepaald bereik voorgesteld op een ander niet-negatief
domein [Thijssen, 1999].
Dat levert eenvoudige, compacte en snelle hardware op.
De keerzijde is dat het voor mensen niet eenvoudig is om (de waarde van) van
de negatieve getallen direct te herkennen.

Daarnaast levert het gebruik van de 2's complement representatie nog een
probleem op: het resultaat van een bewerking kan buiten het bereik van de
representatie liggen. Er is dan sprake van \textsl{overflow}.

Het detecteren van een overflow is een vereiste om de berekeningen in goede
banen te laten lopen. Gelukkig is het detecteren van overflow geen moeilijke
klus. Natuurlijk is de vraag: wat moet er gebeuren als er
overflow geconstateerd is. Dat is aan de ontwerper van de schakeling.

In de onderstaande paragrafen wordt overflowdetectie bij optellen en
aftrekken besproken.

\section{De 2's complement representatie}
In het dagelijkse leven gebruiken wij de zogenaamde teken-en-grootte
representatie om getallen weer te geven en berekeningen uit te voeren.
Het voordeel van deze representatie is dat in \'e\'en oogopslag de grootte
van het getal te zien is, of het nou positief of negatief is. Zo zijn de
getallen $+13$ en $-13$ even groot maar tegengesteld.

Berekeningen uitvoeren in deze representatie is lastig.
Willen we dit omzetten naar een digitale schakeling, dan levert dat erg
veel hardware op. Beter is om een complement representatie te gebruiken.
Complement representaties zijn gebaseerd op \textsl{modulair rekenen}. In
hardware wordt dat al van nature gerealiseerd omdat er met een eindig aantal
bits wordt gerekend.

Zo zijn met vier bits de getallen 0 t/m 15 ($2^4-1$) te realiseren, er zijn
16 ($2^4$) combinaties. Een optelling van twee getallen van vier bits levert
een antwoord van vijf bits op, maar het vijfde bit wordt niet opgeslagen, er
zijn immers maar vier bits beschikbaar. Alle berekeningen zijn dus
\textsl{modulo 2}$^4$. Hieronder een paar voorbeelden.

\hrulefill
\begin{example}
\label{exa:somemode24examples}
\begin{equation*}
\begin{array}{rcrcr}
(0 + 0)   \mod 2^4  &=&   0 \mod 2^4  &=&   0 \mod 2^4 \\
(10 + 5)  \mod 2^4  &=&  15 \mod 2^4  &=&  15 \mod 2^4 \\
\\                                       
(13 + 9)  \mod 2^4  &=&  22 \mod 2^4  &=&   6 \mod 2^4 \\
(15 + 15) \mod 2^4  &=&  30 \mod 2^4  &=&  14 \mod 2^4 \\
\\                                       
(14 + 15) \mod 2^4  &=&  29 \mod 2^4  &=&  13 \mod 2^4 \\
(13 + 15) \mod 2^4  &=&  28 \mod 2^4  &=&  12 \mod 2^4 \\
(12 + 15) \mod 2^4  &=&  27 \mod 2^4  &=&  11 \mod 2^4
\end{array}
\end{equation*}
\end{example}
\hrulefill

Wat bij de laatste drie voorbeelden opvalt, is dat wanneer er bij een getal~15
wordt opgeteld, het antwoord \'e\'en lager is dan dat getal. We kunnen een
getal dus met \'e\'en verlagen door er 15 bij op te tellen. Merk op dat 15
gelijk is aan $2^4-1$.

%%In de 2's complement representatie wordt een negatief getal voorgesteld door
%%het op te tellen bij een macht van 2. Zo kan het getal $-m$ worden afgebeeld
%%als
%%%
%%\begin{equation}
%%-m \quad\leftrightarrow\quad 2^n - m
%%\end{equation}
%%
%%waarbij $m$ niet willekeurig groot mag zijn.
%%Bij deze representatie worden $n$ bits gebruikt zodat alle berekeningen
%%modulo $2^n$ zijn.

We kunnen dit aantonen door de optelling van twee
$n$-bits getallen $a$ en~$b$:
%
\begin{equation}
\label{eqn:aplusbmod2n}
a + b \equiv a + b + 2^n \equiv a + (2^n + b) \pmod{2^n}
\end{equation}
%
De operator $\equiv$ betekent dat de leden in \eqref{eqn:aplusbmod2n}
\textsl{congruent modulo} zijn.
%
Vervangen we nu $b$ door $-b$ dan volgt uit \eqref{eqn:aplusbmod2n}
%
\begin{equation}
\label{eqn:aminbisaplus2nminb}
a + (-b) \equiv a + (2^n + (- b)) \equiv a + (2^n - b) \pmod{2^n}
\end{equation}
%
We kunnen $-b$ dus representeren door het op te tellen bij een macht van 2:
%
\begin{equation}
\label{equ:minusbto2nminusb}
-b \equiv 2^n + (-b) \equiv 2^n - b \pmod{2^n}
\end{equation}
%
Merk trouwens op dat $b$ niet willekeurig groot mag zijn.
We geven nu twee voorbeelden.

\hrulefill
\begin{example}
\label{exa:convdecto2scomp}
Stel $n = 4$ en $b = -3$. Dan volgt uit \eqref{equ:minusbto2nminusb} dat $-3$
gerepresenteerd kan worden door $2^4-3 = 16 - 3 = 13$.

Stel $n = 8$ en $b = -58$. Dan volgt uit \eqref{equ:minusbto2nminusb} dat $-8$
gerepresenteerd kan worden door $2^8-58 = 256 - 58 = 198$.
\end{example}%
\unskip\hrulefill

%%%De reden voor het gebruik van deze representatie is dat de logische schakeling
%%%voor het realiseren van optellen en aftrekken veel eenvoudiger is dan
%%%bij de bekende teken-en-grootte representatie.

We willen graag met $n$ bits een gelijkmatige verdeling van positieve en
negatieve getallen dus we zoeken naar een balans in het aantal negatieve
en positieve getallen. De meest voor de hand liggende verdeling is om de
heeft van de combinaties voor de positieve getallen te gebruiken en de
andere helft voor de negatieve getallen. We gebruiken \'e\'en bit om het
teken aan te geven zodat er $n-1$ bits overblijven voor de grootte.
Hieruit volgt dat voor het bereik van eerder genoemde getal $b$ geldt
%
\begin{equation}
\label{eqn:bereik2scomp}
-2^{n-1} \leq b \leq 2^{n-1}-1
\end{equation}
%
Te zien is dat de verdeling niet symmetrisch is. Er is \'e\'en meer getal
negatief dan positief. Dat is te verklaren uit het feit dat het getal 0
als positief wordt gezien. Een nadeel is dat voor het getal $-2^{n-1}$ geen
tegengesteld getal gerepresenteerd kan worden, er bestaat dus geen $+2^{n-1}$.
Er kan trouwens voor elke willekeurige verdeling% van de positieve en negatieve
%getallen
gekozen worden zolang er maar $2^n$ mogelijke combinaties te maken
zijn, maar op praktische gronden is gekozen voor de verdeling
in \eqref{eqn:bereik2scomp}.

Stel we nemen twee getallen $B$ en $b$ zodanig dat
%
\begin{equation}
\begin{split}
0 \leq b \leq 2^{n-1}-1 &\;\leftrightarrow\; B = b \\
-2^{n-1} \leq b \leq -1 &\;\leftrightarrow\; B = 2^n + b \\
\end{split}
\end{equation}
%
dan wordt $B$ de \textsl{two's complement representatie} van $b$ genoemd.

Kijken we naar een positief getal en coderen dit met bits, dan zien we
dat het getal met $n$ bits ligt in het bereik $0$ -- $2^{n-1}-1$. Het meest
significante bit een $0$. Een negatief getal wordt geschreven als $2^n - m$
met $m$ in het bereik van $1$ -- $2^{n-1}$ zodat het meest significante bit
een $1$ is. De meest significante bits worden tekenbits genoemd.

%\begin{example}
%\label{exa:nummer1}
%Test
%\end{example}
%
%Zie voorbeeld \ref{exa:nummer1}.

\section{Overflow bij optelling}
Overflow in de 2's complement representatie kan eenvoudig worden gedetecteerd
aan de hand van de te bewerken getallen, het resultaat en de bewerking.

% We
%behandelen eerst de optelling, daarna de aftrekking.

Bij een optelling van twee 2's complement getallen kan een overflow alleen
maar optreden als de twee getallen hetzelfde teken hebben, met de nadruk op
\textsl{kan}. Dat is te zien in de onderstaande vergelijkingen. Hierbij worden
de grenzen van de 2's complement representatie opgezocht. Deze grenzen zijn
$-2^{n-1}$ en $-1$ voor negatieve getallen en $0$ en $2^{n-1}-1$ voor
positieve getallen.
%
\begin{align}
\label{eqn:maybeoverflowsamesigns1}
      -2^{n-1} + -2^{n-1} &= -2^{n} \\
\label{eqn:maybeoverflowsamesigns2}
              (-1) + (-1) &= -2 \\
\label{eqn:maybeoverflowsamesigns3}
                    0 + 0 &= 0 \\
\label{eqn:maybeoverflowsamesigns4}
(2^{n-1}-1) + (2^{n-1}-1) &= 2^n -2
\end{align}
%
De vergelijkingen \ref{eqn:maybeoverflowsamesigns1} en
\ref{eqn:maybeoverflowsamesigns4} leveren een overflow op, de vergelijkingen
\ref{eqn:maybeoverflowsamesigns2} en \ref{eqn:maybeoverflowsamesigns3} doen
dat niet.

Als we twee getallen met verschillend teken optellen kan er nooit overflow
optreden. Ook dat is eenvoudig aan te tonen. We zoeken hiervoor weer de grenzen
van het bereik van de 2's complement representatie op.
%
\begin{align}
\label{eqn:nooverflowdiffsigns1}
        -2^{n-1} + 0 &= -2^{n-1} \\
              -1 + 0 &= -1 \\
-2^{n-1} + (2^{n-1}-1) &= -1 \\
\label{eqn:nooverflowdiffsigns2}
      -1 + (2^{n-1}-1) &= 2^{n-1}-2
\end{align}
%
De antwoorden van vergelijkingen \eqref{eqn:nooverflowdiffsigns1} t/m
\eqref{eqn:nooverflowdiffsigns2} liggen allemaal in het bereik van de 2's
complement representatie.

Laten er vergelijking \eqref{eqn:maybeoverflowsamesigns1} en
\eqref{eqn:maybeoverflowsamesigns2} nog eens bekijken waarbij we de optelling
nu modulo $2^n$ beschouwen, dat wil zeggen dat het antwoord in $n$ bits moet
passen met zoals is gegeven
in \eqref{eqn:bereik2scomp}. Uit deze
vergelijkingen is op te maken dat de optelling van twee negatieve getallen
een antwoord tussen $-2$ en $-2^n$ oplevert. Bij alle antwoorden die kleiner
zijn dan $-2^{n-1}$ is er sprake van overflow. We defini\"eren nu een getal
$p$ zodanig dat
\begin{equation}
-2^{n-1} + p < -2^{n-1} \qquad\qquad \mathrm{(met }-2^{n-1} \leq p \leq -1\mathrm{)}
\end{equation}
%
Dan volgt daaruit
%
%\begin{equation}
%\label{eqn:negoverflow}
%   -2^{n-1} + p \mod 2^n = -2^{n-1} +p +2^{n} \mod 2^n = 2^{n-1} +p \mod 2^n
%\end{equation}
\begin{equation}
\label{eqn:negoverflow}
   -2^{n-1} + p \equiv -2^{n-1} +p +2^{n} \equiv 2^{n-1} +p \pmod{2^n}
\end{equation}
%
Vullen we in \eqref{eqn:negoverflow} alle mogelijke waarden van $p$ in dan
volgt dat bij een overflow bij optelling van twee negatieve getallen het
antwoord positief wordt.

We herhalen het bovenstaande voor de optelling van twee positieve getallen.
Het antwoord van deze optelling ligt tussen $0$ en $2^n-2$. Zie hiervoor de
vergelijkingen \eqref{eqn:maybeoverflowsamesigns3} en
\eqref{eqn:maybeoverflowsamesigns4}. Bij alle antwoorden de groter zijn dan
$2^{n-1}-1$ is er sprake van overflow. We defini\"eren een getal $q$ zodanig dat
\begin{equation}
2^{n-1} -1 + q > 2^{n-1}-1 \qquad\qquad \mathrm{(met\ } 1 \leq q \leq 2^{n-1}-1\mathrm{)}
\end{equation}
%
Dan volgt hieruit
%
%\begin{equation}
%\label{eqn:posoverflow}
%(2^{n-1}-1) + q \mod 2^n = (2^{n-1}-1) + q -2^n \mod 2^n = -2^{n-1} + q \mod 2^n
%\end{equation}
\begin{equation}
\label{eqn:posoverflow}
(2^{n-1}-1) + q \equiv (2^{n-1}-1) + q -2^n \equiv -2^{n-1} -1 + q \pmod{2^n}
\end{equation}
In \eqref{eqn:posoverflow} vullen we alle mogelijke waarden van $q$ in dan
volgt dat bij een overflow bij optelling van twee positieve getallen het
antwoord negatief wordt.

Resumerend kunnen we zeggen dat

\bigskip
\begin{center}
\setlength{\fboxsep}{0.8em}%
\fbox{\begin{minipage}{0.9\textwidth}
\centering\textbf{Overflow treedt op als de tekens van de op te tellen
getallen gelijk zijn en ongelijk zijn aan het teken van het antwoord}
\end{minipage}}
\end{center}
\bigskip

We geven twee voorbeelden waarbij overflow optreedt. We gebruiken hiervoor
getallen van vier bits.

\hrulefill
\begin{example}
In dit voorbeeld worden getallen met gelijk teken opgeteld met
als resultaat dat het teken van het antwoord verandert.
\begin{table}[H]
  \centering
  \begin{tabular}{ r c l p{2cm} r c l}
%  1) 1110 &   & carry's & & 0) 1000 &   & carry's\\
      1001 &   & $-7$    & &    0101 &   & $+5$   \\
      1101 & + & $-3$    & &    0110 & + & $+6$   \\  \cmidrule{1-1}\cmidrule{5-5}
   1) 0110 &   & $-10$?  & & 0) 1011 &   & $+11$? \\
  \end{tabular}
\end{table}
\end{example}
\unskip\hrulefill

%De twee getallen 1) en 0) geven de uitgaande carry's aan.
%Uit de twee bovenstaande voorbeelden is eenvoudig te bepalen of er
%overflow is opgetreden door functie \eqref{eqn:funcoverflowadd} uit
%te werken.
Uit bovenstaande voorbeelden is eenvoudig te bepalen of er overflow is
opgetreden door de tekenbits te vergelijken.

Stel dat we de op te telken getallen $a$ en $b$ noemen met als antwoord getal
$s$ (met $a$, $b$ en $s$ alle $n$ bits getallen), dan kunnen we de detectie
van overflow als volgt in een functie beschrijven:
%
\begin{equation}
\label{eqn:overflowaddhighlevel}
V_{add} = \left\lbrace (a_{n-1} = b_{n-1}) \neq s_{n-1} \right\rbrace
\end{equation}
%
Zetten we deze functie om in een schakelfunctie dan volgt:
%
\begin{equation}
\begin{split}
\label{eqn:funcoverflowadd}
V_{add} &= \oline{a_{n-1}}\cdot\oline{b_{n-1}}\cdot s_{n-1} + a_{n-1}\cdot b_{n-1}\cdot\oline{s_{n-1}}
\end{split}
\end{equation}
%
We bekijken nu de optelling van alleen de tekenbits van de getallen en nemen
daar in de ingaande en uitgaande carry's mee. We kunnen de optelling dan
uitschrijven als:

\begin{table}[H]
  \centering
  \begin{tabular}{ r c l}
           $C_{n-1}$... &   & ingaande carry's \\
           $a_{n-1}$... &   & getal $A$    \\
           $b_{n-1}$... & + & getal $B$   \\  \cmidrule{1-1}
    $C_n$) $s_{n-1}$... &   & uitkomst $S$ en uitgaande carry  \\
  \end{tabular}
\end{table}
\unskip%

Hierin zijn $C_{n-1}$ en $C_n$ resp.\@ de ingaande en uitgaande carry's.
We vullen nu voor $a_{n-1}$, $b_{n-1}$ en $s_{n-1}$ alle mogelijke combinaties
is en berekenen daaruit de waarden voor $C_n$ en $C_{n-1}$.

$a_{n-1}=0$, $b_{n-1}=0$, $s_{n-1}=0$. Uit de optelling van $a_{n-1}$ en
$b_{n-1}$ volgt dat $s_{n-1}=0$. Om deze optelling kloppend te maken moet
$C_{n-1} = 0$ zijn en dan volgt automatisch dat $C_n = 0$.

$a_{n-1}=0$, $b_{n-1}=0$, $s_{n-1}=1$. De optelling van de twee sombits levert
een 1 in het antwoord. Dat kan alleen als $C_{n-1} = 1$ en dan volgt dat
$C_n = 0$. Tevens is er overflow opgetreden.

$a_{n-1}=0$, $b_{n-1}=1$, $s_{n-1}=0$. Om uit de optelling van $0+1$ een
$0$ in het sombit te krijgen, moet $C_{n-1} = 1$ zijn. Dan volgt dat
$C_n = 1$, immers $1+0+1=10$.

$a_{n-1}=0$, $b_{n-1}=1$, $s_{n-1}=1$. Hieruit volgt dan $C_{n-1} = 0$ en
$C_n = 0$.

$a_{n-1}=1$, $b_{n-1}=0$, $s_{n-1}=0$. Deze situatie is vergelijkbaar met
de derde situatie.

$a_{n-1}=1$, $b_{n-1}=0$, $s_{n-1}=1$. Deze situatie is vergelijkbaar met
de vierde situatie.

$a_{n-1}=1$, $b_{n-1}=1$, $s_{n-1}=0$. Het optellen van $1+1$ levert als
sombit een $0$ op, dat kan alleen als $C_{n-1} = 0$. Verder volgt dat 
$C_n = 1$. Tevens is er overflow opgetreden.

$a_{n-1}=1$, $b_{n-1}=1$, $s_{n-1}=1$. Het optellen van $1+1$ levert als
sombit een $1$ op, dat kan alleen als $C_{n-1} = 1$. Verder volgt dat 
$C_n = 1$.


We stellen voor alle combinaties van $a_{n-1}$, $b_{n-1}$ en $s_{n-1}$ een
waarheidstabel op en tonen hierin de carry's en of er overflow optreedt.

\begin{table}[ht!]
	\centering
	\caption{Relaties tussen overflow en de carry's bij optelling van twee getallen}
	\label{tab:ovdetadd}
	\begin{tabular}{cccccccc}
	\toprule
	$a_{n-1}$ & $b_{n-1}$ & $s_{n-1}$ & & $C_n$ & $C_{n-1}$ & & $V$ \\ \midrule
	  0   &   0   &   0   & &   0   &   0   & &  0  \\
	  0   &   0   &   1   & &   0   &   1   & &  1  \\
	  0   &   1   &   0   & &   1   &   1   & &  0  \\
	  0   &   1   &   1   & &   0   &   0   & &  0  \\ \addlinespace[0.75ex]
	  1   &   0   &   0   & &   1   &   1   & &  0  \\
	  1   &   0   &   1   & &   0   &   0   & &  0  \\
	  1   &   1   &   0   & &   1   &   0   & &  1  \\
	  1   &   1   &   1   & &   1   &   1   & &  0  \\ \bottomrule
	\end{tabular}
\end{table}

Uit tabel \ref{tab:ovdetadd} blijkt dat overflow optreedt als $C_nC_{n-1} = 01$
 of $C_nC_{n-1} = 10$.

\begin{equation}
\setlength{\fboxsep}{6pt}
\label{eqn:overflowcncn1}
\boxed{V_{add} = C_n \oplus C_{n-1}}
\end{equation}

Overflow treedt op als bij de op als bij optelling van de tekenbits de ingaande
en uitgaande carry's ongelijk zijn.


\end{document}